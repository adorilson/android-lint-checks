\chapter{Introdução}

\section{Motivação}
O mercado de dispositivos móveis tem crescido muito rapidamente nos últimos anos.
Segundo \cite{lecheta}, estudos mostram que hoje em dia mais de 3 bilhões de pessoas
possuem um aparelho celular. Desenvolvedores de aplicações desejam disponibilizar
seus aplicativos para o máximo número de dispositivos. Dispositivos estes que
possuem diversas diferenças entre si, principalmente considerando smartphones e
tablets: tamanho e qualidade de tela, existência ou não de recursos como telefone
GSM, bluetooth, EDGE, 3G, WiFi, câmera, GPS, bússola, e acelerômetro entre outros.

Um dos fatores que influenciam na escolha de um aparelho é o sistema operacional.
O Android é um dos sistemas operacionais mais utilizados no mundo, estando 
disponível em diversos tipos de aparelhos, com as mais variadas configurações.
O que reforça a necessidade de mecanismos para gerenciar as diversas variações entre 
eles.

Dessa forma, é necessário determinarmos os possíveis pontos de variação da plataforma
Android, assim como entender os mecanismos que a plataforma oferece para auxiliar
o gerenciamento dessas variações. 

Com isso, seremos capazes de criar aplicações que possam ser executadas em dispositivos
com características distintas, e que utilizem esses diferentes recursos de forma otimizada.

A documentação oficial da plataforma disponibiliza alguns guidelines para adequar
as aplicações de forma a atenderem essa necessidade de portabilidade. No entanto, 
a verificação se uma aplicação segue essas guidelines é feita de forma manual, pelos
desenvolvedores, o que pode causar erros. Não existem, ou pelo menos não são
oficialmente recomendados, ferramentas que façam essa verificação de forma automática.

Análise estática é uma técnica que pode ser utilizada para essa
verificação automática. Android Lint faz uso de análise estática para verificar
erros comuns em aplicações e possui uma estrutura extensível, sobre a qual poderia
ser desenvolvidos novas verificações que atendessem à necessidade citada acima

% ------------------------------------------------------------------------------
\section{Limitação dos trabalhos relacionados}


\section{Contribuição do trabalho}
Escrever a contribuição dessa trabalho...

\section{Estrutura do trabalho}
Apresetnar a estrutura do trabalho...
