\begin{landscape}
\appendix
\chapter{Checks predefinidos do Android Lint relacionados a padrões e variabilidades}
\label{apd_cheks_uteis}
\begin{longtable}{p{30mm}|p{180mm}|p{25mm}}
%\caption[Feasible triples for a highly vari                                                                                                                                                                                                                                                                                                                                                                                                                                                                                                                                                                                                                                                                                                                                                                                                                                                                                                                                                                                                                                                                                                                                                                          able Grid]{Feasible triples for highly variable Grid, MLMMH.} \label{grid_mlmmh} \\

\hline \multicolumn{1}{|c|}{\textbf{Check}} & \multicolumn{1}{c|}{\textbf{Descrição}} & \multicolumn{1}{c|}{\textbf{Variabilidade}} \\ \hline 
\endfirsthead

\multicolumn{3}{c}%
{{\bfseries \tablename\ \thetable{} -- continuação da página anterior}} \\
\hline \multicolumn{1}{|c|}{\textbf{Check}} &
\multicolumn{1}{c|}{\textbf{Descrição}} &
\multicolumn{1}{c|}{\textbf{Variabilidade}} \\ \hline 
\endhead

\hline \multicolumn{3}{|r|}{{Continua na próxima página}} \\ \hline
\endfoot

\hline \hline
\endlastfoot

\hline                               
UnusedAttribute & 
This check finds attributes set in XML files that were introduced in a version
newer than the oldest version targeted by your application (with the the minSdkVersion
attribute). This is not an error; the application will simply ignore the attribute.
However, if the attribute is important to the appearance of functionality of your
application, you should consider finding an alternative way to achieve the same
result with only available attributes, and then you can optionally create a copy
of the layout in a layout-vNN folder which will be used on API NN or higher where
you can take advantage of the newer attribute. Note: This check does not only apply
to attributes. For example, some tags can be unused too, such as the new <tag> element
in layouts introduced in API 21.
& Diferentes versões da API \\

Back button & According to the Android Design Guide, "Other platforms use an explicit
back button with label to allow the user to navigate up the application's hierarchy.
Instead, Android uses the main action bar's app icon for hierarchical navigation
and the navigation bar's back button for temporal navigation." This check is not
very sophisticated (it just looks for buttons with the label "Back"), so it is
disabled by default to not trigger on common scenarios like pairs of Back/Next
buttons to paginate through screens. http://developer.android.com/design/patterns/pure-android.html
& Padrão de Design \\

Button order & According to the Android Design Guide, "Action buttons are typically
Cancel and/or OK, with OK indicating the preferred or most likely action. However,
if the options consist of specific actions such as Close or Wait rather than a
confirmation or cancellation of the action described in the content, then all the
buttons should be active verbs. As a rule, the dismissive action of a dialog is
always on the left whereas the affirmative actions are on the right." This check
looks for button bars and buttons which look like cancel buttons, and makes sure
that these are on the left. http://developer.android.com/design/building-blocks/dialogs.html
& Padrão de Design \\

Button should be borderless & Button bars typically use a borderless style for
the buttons. Set the style="?android:attr/buttonBarButtonStyle" attribute on each
of the buttons, and set style="?android:attr/buttonBarStyle" on the parent layout
http://developer.android.com/design/building-blocks/buttons.html
& Padrão de Design \\

Byte order mark inside files & Lint will flag any byte-order-mark (BOM) characters
it finds in the middle of a file. Since we expect files to be encoded with UTF-8
(see the EnforceUTF8 issue), the BOM characters are not necessary, and they are
not handled correctly by all tools. For example, if you have a BOM as part of a
resource name in one particular translation, that name will not be considered
identical to the base resource's name and the translation will not be used.
http://en.wikipedia.org/wiki/Byte\_order\_mark
& No exemplo dado, pode ter influencia na variabilidade de idiomas\\
 
Calling new methods on older versions
&This check scans through all the Android API calls in the application and warns about any calls that are not available on all versions targeted by this application (according to its minimum SDK attribute in the manifest). If you really want to use this API and don't need to support older devices just set the minSdkVersion in your build.gradle or AndroidManifest.xml files. If your code is deliberately accessing newer APIs, and you have ensured (e.g. with conditional execution) that this code will only ever be called on a supported platform, then you can annotate your class or method with the @TargetApi annotation specifying the local minimum SDK to apply, such as @TargetApi(11), such that this check considers 11 rather than your manifest file's minimum SDK as the required API level. If you are deliberately setting android: attributes in style definitions, make sure you place this in a values-vNN folder in order to avoid running into runtime conflicts on certain devices where manufacturers have added custom attributes whose ids conflict with the new ones on later platforms. Similarly, you can use tools:targetApi="11" in an XML file to indicate that the element will only be inflated in an adequate context.
&Diferentes versões da API\\

Cancel/OK dialog button capitalization
&The standard capitalization for OK/Cancel dialogs is "OK" and "Cancel". To ensure that your dialogs use the standard strings, you can use the resource strings @android:string/ok and @android:string/cancel.
&Padrao de Design\\

Clashing PNG and 9-PNG files
&If you accidentally name two separate resources file.png and file.9.png, the image file and the nine patch file will both map to the same drawable resource, @drawable/file, which is probably not what was intended.
&"Imagens que podem ser utilizadas em telas de tamanhos/resoluções diferente
(um referencia: http://www.thiengo.com.br/9-patch-no-android-mantendo-a-qualidade-de-imagens-de-background)"\\

Class is not registered in the manifest
&Activities, services and content providers should be registered in the AndroidManifest.xml file using <activity>, <service> and <provider> tags. If your activity is simply a parent class intended to be subclassed by other "real" activities, make it an abstract class. http://developer.android.com/guide/topics/manifest/manifest-intro.html
&Útil quando “activity is simply a parent class intended to be subclassed by other "real" activities” e cada subclasse será implementação de uma variante\\

Custom views in libraries should use res-auto-namespace
&When using a custom view with custom attributes in a library project, the layout must use the special namespace http://schemas.android.com/apk/res-auto instead of a URI which includes the library project's own package. This will be used to automatically adjust the namespace of the attributes when the library resources are merged into the application project.
&Se o núcleo de uma LPS for definido como um “library project” isso será ser útil\\

Deprecated Gradle Construct
&This detector looks for deprecated Gradle constructs which currently work but will likely stop working in a future update.
&Variabilidade de plataforma, mas em tempo de projeto, não de execução\\

Duplicated icons under different names
&If an icon is repeated under different names, you can consolidate and just use one of the icons and delete the others to make your application smaller. However, duplicated icons usually are not intentional and can sometimes point to icons that were accidentally overwritten or accidentally not updated.
&Pode influenciar em dispositivos com pouco espaço\\

Dynamic text should probably be selectable
&Dynamic text should probably be selectable If a <TextView> is used to display data, the user might want to copy that data and paste it elsewhere. To allow this, the <TextView> should specify android:textIsSelectable="true". This lint check looks for TextViews which are likely to be displaying data: views whose text is set dynamically. This value will be ignored on platforms older than API 11, so it is okay to set it regardless of your minSdkVersion.
&Relacionado com versões diferentes da API\\

Encoding used in resource files is not UTF-8
&XML supports encoding in a wide variety of character sets. However, not all tools handle the XML encoding attribute correctly, and nearly all Android apps use UTF-8, so by using UTF-8 you can protect yourself against subtle bugs when using non-ASCII characters. In particular, the Android Gradle build system will merge resource XML files assuming the resource files are using UTF-8 encoding.
&Variabilidade de plataforma, mas em tempo de projeto, não de execução\\

Extra translation
&If a string appears in a specific language translation file, but there is no corresponding string in the default locale, then this string is probably unused. (It's technically possible that your application is only intended to run in a specific locale, but it's still a good idea to provide a fallback.). Note that these strings can lead to crashes if the string is looked up on any locale not providing a translation, so it's important to clean them up.
&Variabilidade de idioma\\

Formatting argument types incomplete or inconsistent
&When a formatted string takes arguments, it usually needs to reference the same arguments in all translations (or all arguments if there are no translations. There are cases where this is not the case, so this issue is a warning rather than an error by default. However, this usually happens when a language is not translated or updated correctly.
&Variabilidade de idioma\\

Fragment not instantiatable
&From the Fragment documentation: Every fragment must have an empty constructor, so it can be instantiated when restoring its activity's state. It is strongly recommended that subclasses do not have other constructors with parameters, since these constructors will not be called when the fragment is re-instantiated; instead, arguments can be supplied by the caller with setArguments(Bundle) and later retrieved by the Fragment with getArguments(). http://developer.android.com/reference/android/app/Fragment.html\#Fragment()
&Relacionado a tamanho de telas quando o framento é usado pra trata esse tipo de variabilidade\\

Fragments should specify an id or tag
&If you do not specify an android:id or an android:tag attribute on a <fragment> element, then if the activity is restarted (for example for an orientation rotation) you may lose state. From the fragment documentation: "Each fragment requires a unique identifier that the system can use to restore the fragment if the activity is restarted (and which you can use to capture the fragment to perform transactions, such as remove it). * Supply the android:id attribute with a unique ID. * Supply the android:tag attribute with a unique string. If you provide neither of the previous two, the system uses the ID of the container view. http://developer.android.com/guide/components/fragments.html
&Relacionado a tamanho de telas quando o framento é usado pra trata esse tipo de variabilidade\\

Gradle Dynamic Version
&Using + in dependencies lets you automatically pick up the latest available version rather than a specific, named version. However, this is not recommended; your builds are not repeatable; you may have tested with a slightly different version than what the build server used. (Using a dynamic version as the major version number is more problematic than using it in the minor version position.)
&Variabilidade de plataforma, mas em tempo de projeto, não de execução\\

Gradle IDE Support Issues
&Gradle is highly flexible, and there are things you can do in Gradle files which can make it hard or impossible for IDEs to properly handle the project. This lint check looks for constructs that potentially break IDE support.
&Variabilidade de plataforma, mas em tempo de projeto, não de execução\\

Gradle Path Issues
&Gradle build scripts are meant to be cross platform, so file paths use Unix-style path separators (a forward slash) rather than Windows path separators (a backslash). Similarly, to keep projects portable and repeatable, avoid using absolute paths on the system; keep files within the project instead. To share code between projects, consider creating an android-library and an AAR dependency
&Variabilidade de plataforma, mas em tempo de projeto, não de execução\\

Hardcoded Package in Namespace
&In Gradle projects, the actual package used in the final APK can vary; for you can add a .debug package suffix in one version and not the other. Therefore, you should not hardcode the application package in the resource; instead, use the special namespace http://schemas.android.com/apk/res-auto which will cause the tools to figure out the right namespace for the resource regardless of the actual package used during the build.
&Util para gerar versões diferentes da mesma app\\

Hardcoded reference to /sdcard
&Your code should not reference the /sdcard path directly; instead use Environment.getExternalStorageDirectory().getPath(). Similarly, do not reference the /data/data/ path directly; it can vary in multi-user scenarios. Instead, use Context.getFilesDir().getPath(). http://developer.android.com/guide/topics/data/data-storage.html\#filesExternal
&Aparelhos pode ter ou não certos recursos (sdcard, no caso)\\

HashMap can be replaced with SparseArray
&For maps where the keys are of type integer, it's typically more efficient to use the Android SparseArray API. This check identifies scenarios where you might want to consider using SparseArray instead of HashMap for better performance. This is particularly useful when the value types are primitives like ints, where you can use SparseIntArray and avoid auto-boxing the values from int to Integer. If you need to construct a HashMap because you need to call an API outside of your control which requires a Map, you can suppress this warning using for example the @SuppressLint annotation.
&Variantes de implementaçao de “maps”, mas com efeito somente em tempo de projeto, an não ser pela performance de execução\\

Icon appears in both -nodpi and dpi folders
&Bitmaps that appear in drawable-nodpi folders will not be scaled by the Android framework. If a drawable resource of the same name appears both in a -nodpi folder as well as a dpi folder such as drawable-hdpi, then the behavior is ambiguous and probably not intentional. Delete one or the other, or use different names for the icons.
&Telas de tamanhos diferentes\\

Icon colors do not follow the recommended visual style
&Notification icons and Action Bar icons should only white and shades of gray. See the Android Design Guide for more details. Note that the way Lint decides whether an icon is an action bar icon or a notification icon is based on the filename prefix: ic\_menu\_ for action bar icons, ic\_stat\_ for notification icons etc. These correspond to the naming conventions documented in http://developer.android.com/guide/practices/ui\_guidelines/icon\_design.html http://developer.android.com/design/style/iconography.html
&Padrao de Design\\

Icon densities validation
&Icons will look best if a custom version is provided for each of the major screen density classes (low, medium, high, extra high). This lint check identifies icons which do not have complete coverage across the densities. Low density is not really used much anymore, so this check ignores the ldpi density. To force lint to include it, set the environment variable ANDROID\_LINT\_INCLUDE\_LDPI=true. For more information on current density usage, see http://developer.android.com/resources/dashboard/screens.html
&Telas diferentes\\

Icon density-independent size validation
&Checks the all icons which are provided in multiple densities, all compute to roughly the same density-independent pixel (dip) size. This catches errors where images are either placed in the wrong folder, or icons are changed to new sizes but some folders are forgotten.
&Telas diferentes\\

Icon density-independent size validation
&Checks the all icons which are provided in multiple densities, all compute to roughly the same density-independent pixel (dip) size. This catches errors where images are either placed in the wrong folder, or icons are changed to new sizes but some folders are forgotten
&Telas diferentes\\

Icon has incorrect size
&There are predefined sizes (for each density) for launcher icons. You should follow these conventions to make sure your icons fit in with the overall look of the platform.
&Telas diferentes\\

Identical bitmaps across various configurations
&If an icon is provided under different configuration parameters such as drawable-hdpi or -v11, they should typically be different. This detector catches cases where the same icon is provided in different configuration folder which is usually not intentional
&Telas diferentes\\

Image defined in density-independent drawable folder
&The res/drawable folder is intended for density-independent graphics such as shapes defined in XML. For bitmaps, move it to drawable-mdpi and consider providing higher and lower resolution versions in drawable-ldpi, drawable-hdpi and drawable-xhdpi. If the icon really is density independent (for example a solid color) you can place it in drawable-nodpi
&Telas diferentes\\

Implied locale in date format
&"Almost all callers should use getDateInstance(), getDateTimeInstance(), or getTimeInstance() to get a ready-made instance of SimpleDateFormat suitable for the user's locale. The main reason you'd create an instance this class directly is because you need to format/parse a specific machine-readable format, in which case you almost certainly want to explicitly ask for US to
ensure that you get ASCII digits (rather than, say, Arabic digits).
Therefore, you should either use the form of the SimpleDateFormat constructor where you pass in an explicit locale, such as Locale.US, or use one of the get instance methods, or suppress this error if really know what you are doing"
&Internacionalização\\

Implied Quantities
&Plural strings should generally include a \%s or \%d formatting argument. In locales like English, the one quantity only applies to a single value, 1, but that's not true everywhere. For example, in Slovene, the one quantity will apply to 1, 101, 201, 301, and so on. Similarly, there are locales where multiple values match the zero and two quantities.
In these locales, it is usually an error to have a message which does not include a formatting argument (such as '\%d'), since it will not be clear from the grammar what quantity the quantity string is describing
&Internacionalização\\

Incomplete translation
&If an application has more than one locale, then all the strings declared in one language should also be translated in all other languages.  If the string should not be translated, you can add the attribute translatable="false" on the <string> element, or you can define all your non-translatable strings in a resource file called donottranslate.xml. Or, you can ignore the issue with a tools:ignore="MissingTranslation" attribute.  By default this detector allows regions of a language to just provide a subset of the strings and fall back to the standard language strings. You can require all regions to provide a full translation by setting the environment variable ANDROID\_LINT\_COMPLETE\_REGIONS.  You can tell lint (and other tools) which language is the default language in your res/values/ folder by specifying tools:locale="languageCode" for the root <resources> element in your resource file. (The tools prefix refers to the namespace declaration http://schemas.android.com/tools.)
&Variabilidade de Idiomas\\

Inefficient layout weight
&When only a single widget in a LinearLayout defines a weight, it is more efficient to assign a width/height of 0dp to it since it will absorb all the remaining space anyway. With a declared width/height of 0dp it does not have to measure its own size first.
&Telas diferentes\\

Minimum SDK and target SDK attributes not defined
&The manifest should contain a <uses-sdk> element which defines the minimum API Level required for the application to run, as well as the target version (the highest API level you have tested the version for.)
&SDK version\\

Missing commit() calls
&After creating a FragmentTransaction, you typically need to commit it as well
&Uso de fragmentos\\

Missing density folder
&Icons will look best if a custom version is provided for each of the major screen density classes (low, medium, high, extra-high, extra-extra-high). This lint check identifies folders which are missing, such as drawable-hdpi. Low density is not really used much anymore, so this check ignores the ldpi density. To force lint to include it, set the environment variable ANDROID\_LINT\_INCLUDE\_LDPI=true. For more information on current density usage, see http://developer.android.com/resources/dashboard/screens.html
&Telas Diferentes\\

Missing explicit orientation
&The default orientation of a LinearLayout is horizontal. It's pretty easy to believe that the layout is vertical, add multiple children to it, and wonder why only the first child is visible (when the subsequent children are off screen to the right). This lint rule helps pinpoint this issue by warning whenever a LinearLayout is used with an implicit orientation and multiple children.  It also checks for empty LinearLayouts without an orientation attribute that also defines an id attribute. This catches the scenarios where children will be added to the LinearLayout dynamically.
&Telas Diferentes\\

Multiple <uses-sdk> elements in the manifest
&The <uses-sdk> element should appear just once; the tools will not merge the contents of all the elements so if you split up the attributes across multiple elements, only one of them will take effect. To fix this, just merge all the attributes from the various elements into a single <uses-sdk> element.
&Versão da API\\

Not overriding abstract methods on older platforms
&To improve the usability of some APIs, some methods that used to be abstract have been made concrete by adding default implementations. This means that when compiling with new versions of the SDK, your code does not have to override these methods.  However, if your code is also targeting older versions of the platform where these methods were still abstract, the code will crash. You must override all methods that used to be abstract in any versions targeted by your application's minSdkVersion.
&API version\\

Potential Plurals
&This lint check looks for potential errors in internationalization where you have translated a message which involves a quantity and it looks like other parts of the string may need grammatical changes.  For example, rather than something like this: <string name="try\_again">Try again in \%d seconds.</string> you should be using a plural: <plurals name="try\_again"> <item quantity="one">Try again in \%d second</item> <item quantity="other">Try again in \%d seconds</item> </plurals> This will ensure that in other languages the right set of translations are provided for the different quantity classes.  (This check depends on some heuristics, so it may not accurately determine whether a string really should be a quantity. You can use tools:ignore to filter out false positives.
&Internacionalização\\

Right-to-left text compatibility issues
&API 17 adds a textAlignment attribute to specify text alignment. However, if you are supporting older versions than API 17, you must also specify a gravity or layout\_gravity attribute, since older platforms will ignore the textAlignment attribute.
&Versão da API\\

Suspicious 0dp dimension
&Using 0dp as the width in a horizontal LinearLayout with weights is a useful trick to ensure that only the weights (and not the intrinsic sizes) are used when sizing the children.  However, if you use 0dp for the opposite dimension, the view will be invisible. This can happen if you change the orientation of a layout without also flipping the 0dp dimension in all the children.
&Telas diferentes\\
                                                                                                                                                                                                                                                                                                                                                                                                                                                                                                                                                                                                                                                                                                                                                                                                                                            
Suspicious Language/Region Combination
&Android uses the letter codes ISO 639-1 for languages, and the letter codes ISO 3166-1 for the region codes. In many cases, the language code and the country where the language is spoken is the same, but it is also often not the case. For example, while 'se' refers to Sweden, where Swedish is spoken, the language code for Swedish is not se (which refers to the Northern Sami language), the language code is sv. And similarly the region code for sv is El Salvador.  This lint check looks for suspicious language and region combinations, to help catch cases where you've accidentally used the wrong language or region code. Lint knows about the most common regions where a language is spoken, and if a folder combination is not one of these, it is flagged as suspicious.  Note however that it may not be an error: you can theoretically have speakers of any language in any region and want to target that with your resources, so this check is aimed at tracking down likely mistakes, not to enforce a specific set of region and language combinations.
&Internacionalização\\

Target SDK attribute is not targeting latest version
&When your application runs on a version of Android that is more recent than your targetSdkVersion specifies that it has been tested with, various compatibility modes kick in. This ensures that your application continues to work, but it may look out of place. For example, if the targetSdkVersion is less than 14, your app may get an option button in the UI.  To fix this issue, set the targetSdkVersion to the highest available value. Then test your app to make sure everything works correctly. You may want to consult the compatibility notes to see what changes apply to each version you are adding support for: http://developer.android.com/reference/android/os/Build.VERSION\_CODES.html
&Versão da API\\

Unused quantity translations
&Android defines a number of different quantity strings, such as zero, one, few and many. However, many languages do not distinguish grammatically between all these different quantities.  This lint check looks at the quantity strings defined for each translation and flags any quantity strings that are unused (because the language does not make that quantity distinction, and Android will therefore not look it up.). For example, in Chinese, only the other quantity is used, so even if you provide translations for zero and one, these strings will not be returned when getQuantityString() is called, even with 0 or 1.
&Internacionalização\\

Unused resources
&Unused resources make applications larger and slow down builds.
&Capacidade do aparelho (armazenameto e velocidade de conexão, do por exemplo)\\

Using 'px' dimension
&For performance reasons and to keep the code simpler, the Android system uses pixels as the standard unit for expressing dimension or coordinate values. That means that the dimensions of a view are always expressed in the code using pixels, but always based on the current screen density. For instance, if myView.getWidth() returns 10, the view is 10 pixels wide on the current screen, but on a device with a higher density screen, the value returned might be 15. If you use pixel values in your application code to work with bitmaps that are not pre-scaled for the current screen density, you might need to scale the pixel values that you use in your code to match the un-scaled bitmap source. 
&Telas diferentes\\

Using 3-letter Codes
&For compatibility with earlier devices, you should only use 3-letter language and region codes when there is no corresponding 2 letter code.
&Dispositivos diferentes\\

Using dp instead of sp for text sizes
&When setting text sizes, you should normally use sp, or "scale-independent pixels". This is like the dp unit, but it is also scaled by the user's font size preference. It is recommend you use this unit when specifying font sizes, so they will be adjusted for both the screen density and the user's preference.  There are cases where you might need to use dp; typically this happens when the text is in a container with a specific dp-size. This will prevent the text from spilling outside the container. Note however that this means that the user's font size settings are not respected, so consider adjusting the layout itself to be more flexible.
&Telas diferentes\\

Using inlined constants on older versions
&This check scans through all the Android API field references in the application and flags certain constants, such as static final integers and Strings, which were introduced in later versions. These will actually be copied into the class files rather than being referenced, which means that the value is available even when running on older devices. In some cases that's fine, and in other cases it can result in a runtime crash or incorrect behavior. It depends on the context, so consider the code carefully and device whether it's safe and can be suppressed or whether the code needs tbe guarded.  If you really want to use this API and don't need to support older devices just set the minSdkVersion in your build.gradle or AndroidManifest.xml files. If your code is deliberately accessing newer APIs, and you have ensured (e.g. with conditional execution) that this code will only ever be called on a supported platform, then you can annotate your class or method with the @TargetApi annotation specifying the local minimum SDK to apply, such as @TargetApi(11), such that this check considers 11 rather than your manifest file's minimum SDK as the required API level.
&Versão da API\\

Using Wrong AppCompat Method
&When using the appcompat library, there are some methods you should be calling instead of the normal ones; for example, getSupportActionBar() instead of getActionBar(). This lint check looks for calls to the wrong method.
&Versão da API\\

Wrong case for view tag
&Most layout tags, such as <Button>, refer to actual view classes and are therefore capitalized. However, there are exceptions such as <fragment> and <include>. This lint check looks for incorrect capitalizations.
&Uso de fragmentos\\

Wrong locale name
&From the java.util.Locale documentation: "Note that Java uses several deprecated two-letter codes. The Hebrew ("he") language code is rewritten as "iw", Indonesian ("id") as "in", and Yiddish ("yi") as "ji". This rewriting happens even if you construct your own Locale object, not just for instances returned by the various lookup methods.  Because of this, if you add your localized resources in for example values-he they will not be used, since the system will look for values-iw instead.  To work around this, place your resources in a values folder using the deprecated language code instead.
&Internacionalização


\end{longtable}

\end{landscape}
