\chapter{Conclusão}
\label{conclusao}
A plataforma Android oferece diversos mecanismos para o tratamento de variabilidades
em uma aplicacão. A documentação oficial da plataforma dispõe de guidelines de 
padrões de implementação dessas variabilidades, sem no entanto, fornecer uma ferramenta
que possa auxiliar aos desenvolvedores a verificação dessas orientações.

Nesse trabalho, apresentamos o Android Lint como uma alternativa para tal tarefa.
Fizemos o mapeamento das regras preexistentes relacionadas a essas variabilidades
e exploramos a capacidade extensão do Android Lint com novas regras. 

As novas regras criadas auxiliam na tarefa de prover suporte adequado e otimizado
para tablets e handsets. As seguintes verificações são feitas:
\begin{itemize}
  \item{Activities devem ser um container de fragmentos (por exemplo, herdar da
  classe FragmentActivity)}
  \item{Activities devem, de fato, ser composta por framentos}
  \item{Activities devem extender de ActionBarActivity}
  \item{O tema da activity deve ser {\it Theme.AppCompat.Light}}
\end{itemize}

Esse foi um trabalho preliminar, para validar a proposta. Além de ser necessário
algum refinamento nessas novas regras definidas, outras regras devem ser analisadas
e implementadas. Por exemplo, o estilo {\it Theme.AppCompat.Light} pode ser extendido
e essa extensão utilizada nas activities. Essa extensão é prevista pela guideline,
no entanto, na implementação atual ela não é considerada. Outra verificação a ser
implementada diz respeito ao uso de classes/métodos/recursos presentes apenas em
versões mais novas do Android, mas que podem ser substituidas por algo de versões
mais antigas sem grandes prejuízos. Se faz necessário fazer o levantamento desses
elementos e posterior implementação para indicar quando essas diferenças de API
não estão sendo tratadas.

Também é necessário compreender melhor o mecanismo de carregamento das classes no
Android Lint. Devido algum detalhe de configuração, não foi possível fazer todas
essas verificações em outras aplicações, por exemplo Google IO, que foi disponibilizada
pela Google para ser utilizada como exemplo prático das melhores práticas para o
desenvolvimento de aplicações Android
\footnote{http://android-developers.blogspot.it/2014/07/google-io-2014-app-source-code-now.html}
e AntennaPod\footnote{https://github.com/AntennaPod/AntennaPod}, um reprodutor de
podcasts opensource.


