\chapter{Verificação de aderências a guideline}

A plataforma Android suporta uma grande variedades de telas e o sistema redesenha
as interfaces das aplicações de forma a preencher cada uma delas. Tipicamente, 
tudo que o desenvolvedor precisa fazer é projetar a interface para ser flexível 
e otimizar alguns elementos para telas diferentes fornecendo recursos alternativos, 
como layouts alternativos que reposicionam e/ou redefinem dimensões de alguns 
elementos. A fim de facilitar esse trabalho, no Android 3.0 (API Level 11) foi 
introduzida uma nova API: Fragments. Fragmentos permitem separar partes de uma 
interface em componentes isolados, que poderão ser combinados para criar interfaces 
com múltiplos painéis para um tablet ou “activities” separadas para um handset. 
Android 3.0 também introduziu outra API: ActionBar. ActionBar fornece um componente 
no topo da tela para identificar a aplicação e provê ações do usuário e navegação.

A documentação oficial da plataforma fornece guidelines para ajudar desenvolvedores
nessa tarefa de otimizar as aplicações para tablets e handsets\footnote{Disponível 
em http://developer.android.com/guide/practices/tablets-and-handsets.html\#Guidelines}:

\begin{itemize}
    \item{Construir a aplicação baseada em fragmentos que possam ser reutilizados 
    em diferentes combinações;}
    \item{Utilizar ActionBar, mas seguir as melhores práticas e certificar-se que 
    o design da aplicação é flexível o suficiente para o sistema ajustar o layout 
    da action bar baseada no tamanho da tela;}
    \item{Implementar layouts flexíveis.}
\end{itemize}

Neste trabalho, apresentamos uma proposta para automatizar a verificação dos dois 
primeiros itens desse guia. Quanto ao terceiro item, usar fragmentos é parte da
implementação de layouts flexíveis.

\section{Uso da API Fragments}
\label{uso_fragments}
Para a nossa abordagem, definimos que os seguintes items devem ser verificados: 
\begin{itemize}
    \item{As activities deverão ser um “container” de fragmentos;}
    \item{A activity deve, de fato, estar sendo composta por fragmentos}
\end{itemize}


A partir da API Level 11, a classe {\it Activity} já suporta nativamente ser composta
por fragmentos. No entanto, na nossa abordagem iremos considerar que as activies
da aplicação deverão extender, de forma direta ou indireta, da classe FragmentActivity,
do pacote de compatibilidade V4. Assim, também garantiremos que a aplicação dará
suporte para aparelhos com versão mais antigas da API.

Para essa primeira verificação foi definido o Issue MainActivityIsFragmentActivity,
conforme figura \ref{mainactIsFragAct}.

\begin{figure}[h]
    \centering
    \includegraphics[width=15cm]{img/mainactIsFragAct.png}
    \caption{Definição do issue MainActivityIsFragmentActivity}
    \label{mainactIsFragAct}
\end{figure}

Caso o detector verifique que a activity não esteja fazendo a herança esperada, 
irá reportar, conforme figura \ref{activity_deve_ser_container}

\begin{figure}[h]
    \centering
    \includegraphics[width=15cm]{img/activity_deve_ser_container.png}
    \caption{Aviso de que a activity deve herdar de FragmentActivity}
    \label{activity_deve_ser_container}
\end{figure}

Para resolver esse problema, basta seguir a orientação apresentada no relatório
e fazer com que a activity {\it MMUnBActivity} herde da classe
{\it android.support.v4.app.FragmentActivity}, conforme apresentado na figura
\ref{herda_FragmentActivity}

\begin{figure}[h]
    \centering
    \includegraphics[width=15cm]{img/heranca_FragmentActivity.png}
    \caption{Activity corretamente extendendo de FragmentActivity}
    \label{herda_FragmentActivity}
\end{figure}

Além de ser um container de fragmentos, é também necessário que a activity seja,
de fato, composto por eles. Existem duas formas de fazer essa composição:
declarando o fragmento no arquivo XML de layout da activity ou programaticamente
adicionando o fragmento na activity. Nessa proposta inicial, foi considerada
somente a forma programatica.

Para fazer a inclusão de forma programatica, um determinado conjunto de métodos
deverá ser executado: {\it getSupportFragmentManager} para obter um objeto {\it
FragmentManager}. Então chamar {\it beginTransaction()} para criar um {\it
FragmentTransaction} e, por fim, {\it add()} para adicionar o fragmento. Também
deve ser executado {\it commit()} quando realizamos múltiplas transações de 
fragmentos com o mesmo {\it FragmentTransaction}. Um exemplo disso pode ser visto
na figura \ref{adicao_fragmento}, que apresenta o método {\it onCreate} da classe
{\it MMUnBActivity}.

\begin{figure}[h]
    \centering
    \includegraphics[width=15cm]{img/add_fragment.png}
    \caption{Adicionando framentos de forma programática}
    \label{adicao_fragmento}
\end{figure}

Embora vários métodos sejam utilizados, nessa abordagem verificamos apenas se o
o método {\it beginTransaction()} do {\it FragmentManager} é chamado a partir do
{\it onCreate()} da activity, inclusive de forma recursiva. Empiricamente,
acreditamos que se esse método está sendo chamado, os demais necessários para a
adição do correta do(s) fragmento(s) a activity também estão. Essa análise é feita
para cada classe da aplicação que estende {\it FragmentActivity}.

\section{Uso da API ActionBar}

A partir do Android 3.0 (API Level 11), a action bar é incluida em todas as
activities que usam o tema Theme.Holo (ou um dos seus descendentes), e que é o
tema padrão quando os atributos {\it targetSdkVersion} ou {\it minSdkVersion}
estão definidos para "11" ou maior.

Para versões anteriores, é possível utilizar o pacote de compatibilidade V7.
Nesse caso, serão necessários dois passos adicionais:

\begin{itemize}
  \item{A activity deve estender ActionBarActivity;}
  \item{A activity deve usar um tema da família Theme.AppCompat, como o
  Theme.AppCompat.Light}
\end{itemize}

Para o primeiro item, a verificação e feita de forma semelhante ao apresentação
na seção \ref{uso_fragments}. Inclusive ActionBarActivity herda de FragmentActivity.
Assim, essa verificação não inválida a descrita naquela seção.

Quando uma activity atende ao primeiro item, o segundo é analisado. Para tal,
é recuperado do arquivo de manifesto o elemento referente à activity e então
o atributo {\it theme} é comparado com {\it Theme.AppCompat.Light}, reportando
um erro caso seja diferente.

